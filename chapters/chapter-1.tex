\chapter{Introduction} \label{ch:introduction}

In a world ravaged by air pollution and climate change, producing clean and safe energy becomes ever more vital. Nuclear energy can fill the void of coal and natural gas, bringing both the clean and safe energy that our world requires. In 2018, the world produced over 2500 TWh of nuclear energy. While this is only a small percentage of the worlds total energy production, it has been steadily growing since 1970. A large part of the commercial success, longevity, safety benefits and efficiency for these reactors has been optimized from the development of modeling and simulation tools. Furthermore, the majority of these commercial nuclear reactors operate based on designs that are decades-old and are being phased out of commission. The next generation of nuclear reactors brings with it added economic promise, increased efficiency, improved fuel management and additional safety benefits. To aid in the design, operation and licensing of these advanced reactors, the existing modeling tools must be modified and new simulation tools need to be created. 
 
Liquid fuel molten salt reactors (MSRs) are a class of next-generation advanced nuclear reactors with great promise and also previous operational history. By design, MSRs operate in a much different way than traditional nuclear reactors. While traditional nuclear reactors work with solid and static manufactured fuel elements, MSRs dissolve the nuclear fuel into a molten salt that continuously flows throughout the reactor's primary loop. This allows for various chemical and isotopic species to transport and react in the loop. During operation of a nuclear reactor, the material composition is constantly changing. These changes come from the neutron irradiation, nuclear decay and chemical reactions. Understanding how the composition changes is key to many other physical processes which occur in nuclear reactors. Modeling these composition changes also gives insight into fuel performance and safety. 

Modeling the composition changes in nuclear reactors is referred to as nuclear fuel depletion calculations. These calculations model the atomic density of various isotopes in a nuclear reactor over long and short time periods. Modeling shorter time periods is required for postulated accident scenarios as well as for rapid changes in neutron precursor concentrations. Long depletion steps are important for understanding fuel burnup, economics and for optimizing fuel performance. In traditional nuclear reactors, these calculations involve solving a large system of stiff first order ordinary differential equations. Using modern matrix exponential methods, the problem of accurately solving these equations has been solved. The problem, however, is that the existing depletion codes where written to model the current fleet of reactors. These current codes do not characterize the underlying physical phenomena that are occurring in advanced reactors with flowing fuel. 

The goal of this dissertation is to start over from scratch, and to redefine the way in which nuclear depletion calculations are done in advanced reactors. Using this ideology, the depletion equations are developed from the fundamental chemical engineering phenomena of mass transport. Starting from the chemical species transport equation, the author has developed a software to model fuel depletion in MSRs. This C++ software is entirely written by the author and is built using the linear algebra library Eigen \cite{eigen}. The software developed is named after one of the author's most favorite movies of all time, "The Big Lebowski," and is herein called ``\textit{libowski}.'' This code is available on GitHub at \href{https://github.com/rtaylo45/libowski}{{\color{blue}https://github.com/rtaylo45/libowski}}.

\section{Outline and Research Goals}

The goal of this work is to generalize fuel depletion calculations for a flowing fueled nuclear reactor. Not only to set up the mathematical expressions to model the multi-physics phenomena but to also develop robust and accurate solutions to these equations. This work is laid out in 6 chapters. Chapter 2 introduces the nuclear burnup equations and works to develop the equations required to model depletion in traditional nuclear reactors and MSRs. Chapter 3 discusses the mathematical framework used to solve the MSR depletion equations. Chapter 4 applies these methods and their practical application in libowski. Chapter 5 shows a set of eight progression problems and five case studies. Finally chapter 6 summarizes the conclusions. 
